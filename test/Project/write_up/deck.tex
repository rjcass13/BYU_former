\documentclass{beamer}

% Preamble settings for customization
\usetheme{Madrid} % A popular theme for a clean look
\usecolortheme{dolphin} % A color scheme to pair with the theme
\usepackage{graphicx}
\usepackage{amsmath}
\usepackage{lmodern}
\usepackage{animate}
\usepackage[backend=biber, style=ieee]{biblatex} 
\addbibresource{refs.bib} 
\renewcommand*{\mkbibacro}[1]{#1} % To avoid an error where it was trying to use an unavailable font size in bib

% Presentation metadata
\title{2D Ising Model Simulation in R}
\author{RJ Cass}
\institute{BYU}
\date{5 Dec. 2025}

\begin{document}

% Title slide
\begin{frame}
    \titlepage
\end{frame}

% Table of Contents slide
\begin{frame}
    \frametitle{Outline}
    \tableofcontents
\end{frame}

\section{Context}

    \begin{frame}{Background}
        \frametitle{What is an Ising model?}
        \begin{itemize}
            \item Array of nodes where each node can be in one of two states (usually represented as +1 and -1)
            \item Core concept is that each node is influenced by its neighbors, and larger system-wide factors
            \item Used for modeling in a variety of fields, such as material sciences, epidemiology, even sociology
            \item As N grows, quickly expands past the bounds of analytical solutions
        \end{itemize} 
        
        \begin{figure}[h!]
            \centering
            \includegraphics[scale=.3]{images/ising_model_arrows.png}
            \caption{Representation of a 2D Ising Model \cite{Ta2o_2025}.}
            \label{fig:ising_arrows}
        \end{figure}
    \end{frame}

    \begin{frame}{Ferromagnetism}
        In ferromagnets, each dipole (node) has a spin (magnetic moment: positive/negative). This alignment is affected by two factors:
        \begin{itemize}
            \item Orientation of Neighbors: dipoles tend to align with their neighbors to reduce entropy of system
            \item Temperature: the external energy from temperature can excite the dipoles and cause their spin to flip spontanesouly
            \begin{itemize}
                \item There is a Curie Temperature ($T_C$) above which the energy of the system overwhelms the 'neighborly influence', resulting in random alignment
            \end{itemize} 
        \end{itemize} 

        The behavior of a ferromagnetic material can be solved analytically in relatively simple scenarios, but becomes virtually unsolvable for more complex scenarios. 
    \end{frame}

    \begin{frame}{Goal}
        \frametitle{Goal/Approach}
        
        Goal: We want to better understand the properties of ferromagnetic materials near the Curie Temperature. \\~\\

        Approach: We will use Markov chain Monte Carlo to imitate a ferromagnetic material's behavior over a range of temperatures and investigate its properties.\\~\\

        Note: For this project I am not using actual scientific values (ie. scaling things differently than real-world scenarios). This is not a rigorous study of a real scenario: it's more like a proof of concept of the process.

    \end{frame}


% Section 2

\section{Mathematics}

    \begin{frame}{Node Equations}

        \textbf{Hamiltonian} \cite[Eq. 2]{Kotze_2008} - influence of the neighbors on a node
        \begin{equation}
            H_i = -J \sum_{j = 1}^{4} s_i s_j
            \label{eq:Hamiltonian}
        \end{equation}
        \begin{itemize}
            \item $J$: coupling constant (+1: ferromagnetic, -1: antiferromagnetic)
            \item $s_i$: spin at the node of interest
            \item $s_j$: spin of the nearest neighbors (in 2D there are 4: above, below, left, right)
        \end{itemize}
        
        \textbf{Metropolis Ratio} \cite[Eq. 12]{Kotze_2008} - influence of temperature and neighbors
        \begin{equation}
            \frac{W(x_i \to x_{i'})}{W(x_{i'} \to x_i)} = \left\{
                \begin{array}{ll}
                e^{-\delta H_i \beta}, & \delta H_i < 0\\
                1,         & \text{otherwise}
                \end{array}
            \right.
            \label{eq:MHRatio}
        \end{equation}
        \begin{itemize}
            \item $\delta H_i \ge 0 \implies e^{-\delta H_i \beta} \ge 1$, hence the acceptance ratio is just 1
            \item $\beta = \frac{1}{k_b T}$, $k_b$ is Boltzman's constant, $T$ is Temperature. For this project I am using $k_b = 1$ and $~1 \le T \le 5$ (unitless)
        \end{itemize}
    \end{frame}

    \begin{frame}{Lattice Equations}
        \textbf{Energy} \cite[Eq. 18]{Kotze_2008}
        \begin{equation}
            \langle E \rangle = \frac{1}{2} \sum_{i = 1}^{N} H_i
            \label{eq:Energy}
        \end{equation}
        
        \textbf{Magnetization} \cite[Eq. 17]{Kotze_2008}
        \begin{equation}
            \langle M \rangle = \frac{1}{N} \sum_{i = 1}^{N} s_i
            \label{eq:Magnetization}
        \end{equation}
        
        \textbf{Specific Heat} \cite[Eq. 19]{Kotze_2008} - How much energy is required to make the elements more 'active' 
        \begin{equation}
            C = \frac{\beta}{T}(\langle E^2 \rangle - \langle E \rangle^2)
            \label{eq:SpecHeat}
        \end{equation}

        \textbf{Susceptibility} \cite[Eq. 20]{Kotze_2008} - How responsive the material is to an applied magnetic field
        \begin{equation}
            \chi = \beta(\langle M^2 \rangle - \langle M \rangle^2)
            \label{eq:Sus}
        \end{equation}
    \end{frame}
    
\section{Process}
    \begin{frame}{Process}
        \frametitle{Process Outline}

        One MCMC step requires, for each element in a given lattice:
        \begin{enumerate}
            \item Determine the Hamiltonian for that element as shown in Equation \eqref{eq:Hamiltonian}
            \item Calculate the acceptance ratio shown in Equation \eqref{eq:MHRatio}
            \item Flip the spin ($s_i = -s_i$) if \textit{runif(1)} is less than the acceptance ratio\\~\\
        \end{enumerate}
        
        The overall process is:
        \begin{enumerate}
            \item Get initial temperature, initialize lattice (random node alignments)
            \item Perform a number of MCMC steps until the lattice reaches equilibrium (effectively the burn-in)
            \item Perform a number of MCMC steps, recording the Energy and Magnetization values at each step 
            \item Calculate $\langle E \rangle, \langle M \rangle, C, \chi$, MCE for that temperature value
            \item Repeat steps 1-4 for each temperature value \\~\\
        \end{enumerate}

        Note: One MCMC 'unit' is the full lattice state, which encapsulates the arrangement of all the nodes
    \end{frame}

    \begin{frame}{Code}
        \frametitle{Brief Coding Overview}

        I attempted to make my code reuseable, defining functions with sensible parameters that could easily be adapted. The Python code provided by Singh \cite{Singh} offered a nice framework as a launching point for setup. 
        \begin{enumerate}
            \item create\_array(): Generates the initial lattice
            \item calc\_ham(): Calculates the Hamiltonian for a given node
            \item calc\_energy(): Calculates the energy of the lattice
            \item calc\_mag(): Calculates the magnetization of the lattice
            \item mc\_step(): Performs one MCMC step (updates lattice to next state)
            \item ising\_sim(): For a given temperature sequence, performs MCMC simulation at each temperature and records the metrics at that temperature
            \item ising\_sim\_plots(): For a given temperature, perform MCMC simulation and plot the lattice at certain points (for display purposes)
            \item ising\_sim\_mag(): For a given temperature, perform MCMC simulation and plot the magnetization per step
        \end{enumerate}
    
    \end{frame}

\section{Results}
    \begin{frame}{Temperature Distributions}
        \begin{figure}[h!]
            \begin{minipage}[b]{0.3\linewidth}
                \centering
                \includegraphics[width=\textwidth]{images/sim_energy.png}
                \caption{Energy}
                \label{fig:sim_energy}
            \end{minipage}
            \begin{minipage}[b]{0.3\linewidth}
                \centering
                \includegraphics[width=\textwidth]{images/sim_mag.png}
                \caption{Magnetization}
                \label{fig:sim_mag}
            \end{minipage}
            \begin{minipage}[b]{0.3\linewidth}
                \centering
                \includegraphics[width=\textwidth]{images/sim_spec_heat.png}
                \caption{Specific Heat}
                \label{fig:sim_spec_heat}
            \end{minipage}
            \begin{minipage}[b]{0.3\linewidth}
                \centering
                \includegraphics[width=\textwidth]{images/sim_sus.png}
                \caption{Susceptibility}
                \label{fig:sim_sus}
            \end{minipage}
            \begin{minipage}[b]{0.3\linewidth}
                \centering
                \includegraphics[width=\textwidth]{images/sim_mce_eng.png}
                \caption{MCE Energy}
                \label{fig:sim_mce_eng}
            \end{minipage}
            \begin{minipage}[b]{0.3\linewidth}
                \centering
                \includegraphics[width=\textwidth]{images/sim_mce_mag.png}
                \caption{MCE Mag.}
                \label{fig:sim_mce_mag}
            \end{minipage}
        \end{figure}
    \end{frame}

    % Uncomment this for final build. Leave it commented for faster compile
    \begin{frame}{Behavior at different temperatures}
        \begin{figure}[h!]
            \begin{minipage}[b]{0.3\linewidth}
                \centering
                \animategraphics[autoplay, loop, width=\textwidth]{4}{temp_1_5_gif/plot_frame_}{1}{45}
                \caption{Temperature = 1.5}
                \label{fig:temp1.5_gif}
            \end{minipage}
            \begin{minipage}[b]{0.3\linewidth}
                \centering
                \animategraphics[autoplay, loop, width=\textwidth]{4}{temp_2_gif/plot_frame_}{1}{45}
                \caption{Temperature = 2}
                \label{fig:temp2_gif}
            \end{minipage}
            \begin{minipage}[b]{0.3\linewidth}
                \centering
                \animategraphics[autoplay, loop, width=\textwidth]{4}{temp_4_gif/plot_frame_}{1}{45}
                \caption{Temperature = 4}
                \label{fig:temp4_gif}
            \end{minipage}
        \end{figure}
    \end{frame}

    % Uncomment this for final build. Leave it commented for faster compile
    \begin{frame}{Magnetization Phase Change}
        \begin{figure}[h!]
            \begin{minipage}[b]{0.45\linewidth}
                \centering
                \includegraphics[width=\textwidth]{images/sim_phase_change.png}
                \caption{Magnetization at each simulation step showing spontaneous phase change}
                \label{fig:phase_change}
            \end{minipage}
            \begin{minipage}[b]{0.45\linewidth}
                \centering
                \animategraphics[autoplay, loop, width=\textwidth]{4}{phase_change_gif/plot_frame_}{1}{101}
                \caption{Animation showing spontaneous phase change}
                \label{fig:phase_change_gif}
            \end{minipage}
        \end{figure}
    \end{frame}


\section{Conclusion}
    \begin{frame}{Next Steps - Code}
        There are a few adjustments that would be good for making my code more re-usable
        \begin{itemize}
            \item Make the function used for the Metropolis Ratio dynamic, allowing a user to define their own and pass it in
            \item Make the code for plot generation more robust (for both the final MCMC plots, and the lattice plots)
            \item Optimize, find a way to make it run fast enough to operate on grids that are magnitudes larger than what I tested
        \end{itemize}

        There are a few things that, using what I currently have, would be interesting next steps
        \begin{itemize}
            \item Adapt the code to use actual material properties (ie. use real value of $k_b$, $T$, etc.)
            \item This can be expanded into 3D applications generally (the Metropolis ratio is much more complex in my scenario, but possible)
            \item An actual field of study: how to make this work not just looking at nearest neighbor, but have the Hamiltonian depend on full material (Dr. Page)
        \end{itemize}
    \end{frame}

\section{Appendix}
    \begin{frame}{Bibliography}
        \frametitle{Bibliography}

        \printbibliography

    \end{frame}

\end{document}