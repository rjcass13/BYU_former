\documentclass{beamer}

% Preamble settings for customization
\usetheme{Madrid} % A popular theme for a clean look
\usecolortheme{dolphin} % A color scheme to pair with the theme
\usepackage{graphicx}
\usepackage{amsmath}

% Presentation metadata
\title{Welch's T-Test}
\author{RJ Cass}
\institute{BYU}
\date{13 Sep. 2025}

% logo of my university
\titlegraphic{\centering\includegraphics[width=0.2\textwidth]{BYU_logo.png}}


\begin{document}

% Title slide
\begin{frame}
    \titlepage
\end{frame}

% Table of Contents slide
\begin{frame}
    \frametitle{Outline}
    
    \tableofcontents
    
\end{frame}

% Section 1
\section{Context}

\begin{frame}{What existed previously?}

    Student's T-Test  
    $$
        u = \frac{(\bar{x}_1 - \bar{x}_2)}{\sqrt{\frac{\Sigma_1 + \Sigma_2}{n_1 + n_2 - 2}(\frac{1}{n_1}+\frac{1}{n_2})}}
    $$

    This test gives the following density for various degrees of freedom
    
    \includegraphics[scale=.3]{Student_t_pdf.svg.png}


\end{frame}

\begin{frame}{Issues of Existing Analyses}
    Issues: 
    \begin{itemize}
        \item Requires assumption of equal variances
        \item Cannot be generalized beyond equal variance scenarios
    \end{itemize} 
\end{frame}

% Section 2
\section{Welch's Approach}

\begin{frame}{Equations}
    \frametitle{Equations}
    
    $$
        v = \frac{(\bar{x}_1 - \bar{x}_2)}{\sqrt{\frac{\Sigma_1}{n_1(n_1 - 1)}+\frac{\Sigma_2}{n_2(n_2 - 1)}}}
    $$
    
\end{frame}

\begin{frame}{Why use Welch's}
    \frametitle{Why use Welch's}

    \begin{itemize}
        \item Can generalize for unequal variances
        \item Less liable to bias
    \end{itemize}

\end{frame}

\end{document}